\chapter*{Abstract}

\mquote{I would have made the book shorter, but I did not have any more time.}{E. H. Connell, Elements of Abstract and Linear Algebra, 1999}
%http://www.math.miami.edu/~ec/book/,

\noindent \textit{Mobile device software applications} play an ever increasing role in the interaction with \textit{ubiquitous computing} environments. More and larger mobile applications need to be implemented and more code needs to be debugged. This brings the need for automated \textit{product-lines} to develop mobile applications. To grow the acceptance of the mobile product-lines among the developers, the technology used to implement them should be easy to implement and introduce. 

There are many \textit{variability mechanisms} that have been proposed over the years to deal with the organization of the common domain functionality in a product-line. Some of these mechanisms, e.g., \textit{component libraries}, are easy to implement, but introduce a lot of accidental complexity in an application. Other variability mechanisms, such as, code generation based on \textit{visual models}, support a more abstract representation of a domain, but do not preserve the architecture of the domain abstractions in the application code. Other mechanisms, e.g., \textit{domain-specific languages} (DSL), are both declarative and preserve well the architecture of the product-line in the source code. Despite the work done in the previous years, DSL remain expensive to introduce and maintain. The variability mechanisms for mobile product-lines should be declarative, easy to introduce, and enable various domains-specific optimizations.

This dissertation addresses mobile product-line development related issues in three dimensions. First, based on the success of the \textit{software container} component technology for server-side enterprise applications, a software container abstraction for mobile applications, called a \textit{mobile container}, is introduced to organize the common functionality of mobile product-lines. Mobile containers serve as a central point to organize the functionality needed by the mobile applications. A mobile container is defined as a \textit{hybrid} client- and server-side container. A mobile application communicates with the rest of a ubiquitous computing environment as a client of one or more server-side services. Mobile containers organize the common functionality of a mobile product-line as a set of services used by mobile clients, and by server-side applications. Server-side applications contain mobile client specific code to adapt the data and the services requested by the mobile clients based on the limitations that exist in mobile devices, e.g., low screen resolution. 

Second, \textit{attributes} are used to model the container services declaratively in the source code. Attributes are known from technologies, such as .NET and Java, where they are used to decorate the program elements of the language, e.g., classes and methods. Attributes extend the language without having to maintain its compiler and are used to emulate \textit{domain-specific abstractions} (DSA). \textit{Attribute families} are used as a means to model the domain variability. A \textit{lightweight} language extension is created that enables accessing and manipulating the annotated code. Languages that contain attribute transformation support, as part of their core functionality, will be called \textit{Generalized and Annotated Abstract Syntax Tree (GAAST)} languages. GAAST-languages enable better language support for the code transformations applied in the context of the \textit{Model-Driven Architecture (MDA)}.

Third, a structured modularization of attribute-driven transformers is introduced to ease the interpretation of attribute-based DSA. \textit{Specialized attribute transformers} connect the attribute-based DSA with the container services, exposed as object-oriented (OO) libraries. Attribute-driven transformers are \textit{horizontally} modularized based on the characteristics of the modeled domain. The semantics of the domain assets, expressed as attribute families, are used to determine the transformation workflow. A uniform composition model for transformers based on the \textit{inner attributes} is developed. Transformers state their dependencies declaratively by decorating the processed elements with arbitrary information in the form of inner attributes. This information is used to coordinate the code processing by successor transformers. The \textit{hardwired} OO language meta-model is utilized to introduce a \textit{vertical} modularization of the attribute-driven transformers. The \textit{transformation strategy} is structured according to the nesting of the structural elements found in the language meta-model. The \textit{layering} strategy is enforced by operations, specialized for every supported element of the language meta-model. These operations form a specialized language for attribute-driven transformations. Common attribute operations, such as, checking for attribute dependencies, are declaratively specified and factored out of the individual transformers by using \textit{meta-attributes}. Meta-attribute enable the decoration of attributes themselves with arbitrary semantics that can be automatically processed.

The proposed technology is evaluated by an \textit{extensible mobile container framework} for Mobile Information Device Profile (MIDP) applications of the Java 2 Mobile Edition (J2ME), that run in small mobile devices. Several technical MIDP concerns are addressed, e.g., data persistence, session and screen management, and networking, resulting in a simpler and more declarative programming model, that preserves the architecture of the domain in the individual applications.
