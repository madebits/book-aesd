\chapter*{Zusammenfassung}

\noindent Anwendungen f\"{u}r mobile Ger\"{a}te spielen eine zunehmende Rolle in  \textit{ubiquitous computing} Umgebungen. Mehr und gr\"{o}{\ss}ere Anwendungen m\"{u}ssen entwickelt und mehr Code muss von Fehlern befreit werden. Dies erh\"{o}ht den Bedarf f\"{u}r automatisierte Produktlinien, um mobile An\-we\-ndun\-gen zu entwickeln. Um die Akzeptanz f\"{u}r mobile Produktlinien unter Entwicklern zu f\"{o}rdern, sollte die f\"{u}r die Entwicklung verwendete Technologie schnell und einfach einzuf\"{u}hren sein.

\"{U}ber die Jahre wurde eine Vielzahl von Variabilit\"{a}tsmechanismem vorgeschlagen, 
um in der Lage zu sein, Funktionalit\"{a}t innerhalb einer Dom\"{a}ne in Produktlinien 
zu organisieren. Einige dieser Mechanismen, z.B. Komponentenbibliotheken sind einfach und schnell einzuf\"{u}hren, bringen aber unbeabsichtigte Komplexit\"{a}t in die Anwendung. Andere Mechanismen die Variabilit\"{a}t bieten, z.B. Code Generierung basierend auf visuellen Modellen, unterst\"{u}tzen zwar eine bessere abstraktere Darstellung der Dom\"{a}ne, behalten aber die Abstraktionen des Designs nicht im Code bei. Wieder andere Mechanismen, z.B. dom\"{a}nenspezifische Sprachen (domain specific languages, DSL) sind deklarativ und  erhalten die Architektur der Produktlinie im Quellcode der An\-we\-ndung. Trotz der Arbeit der vergangen Jahre bleiben DSLs aber mit hohen Einf\"{u}hrungskosten und Wartungskosten verbunden. Der Variabilit\"{a}tsmechanismus f\"{u}r mobile Produktlinien sollte deklarativ sein, einfach einzuf\"{u}hren sein und dom\"{a}nenspezifische Optimierungen unterst\"{u}tzen.

Diese Dissertation befasst sich mit den Problemen der Entwicklung  
mobiler Pro\-dukt\-li\-nien in dreierlei Hinsicht: Erstens, basierend auf dem Erfolg Container basierter Komponententechnologien f\"{u}r serverseitig Unternehmensanwendungen wurden mobile Container definiert. Mobile Container werden genutzt, um das gemeinsame Verhalten von mobilen Produktlinien zu organisieren. 
Mobile Container dienen als zentraler Punkt, um das von mobilen An\-we\-ndu\-ngen ben\"{o}tigte Verhalten zu organisieren und als Container Dienste den Anwendungen transparent zur Verf\"{u}gung zu stellen.
Mobile Anwendungen m\"{u}ssen mit dem Rest einer \textit{ubiquitous computing}   
Umgebung als Client mehrerer Serverdienste kommunizieren. Serverseitige
Dienste ben\"{o}tigen allerdings spezifischen Code f\"{u}r die mobilen Clients, um  die Daten an die begrenzten F\"{a}higkeiten der mobilen Endger\"{a}te anzupassen. Ein mobiler Container ist deswegen eine hybride Einheit, welche aus einem clientseitigen und einem serverseitigen Komponente besteht. Er organisiert das gemeinsame Verhalten, das von Anwendungen auf mobilen Ger\"{a}ten ben\"{o}tigt wird, als Dienste, die auf dem mobilen Ger\"{a}t oder auf Serverseite ausgef\"{u}hrt werden. 

Zweitens, Attribute bieten eine einfache M\"{o}glichkeit, um  Containerdienste deklarativ im Quellcode  anzufordern. Attribute sind aus Technologien
wie .NET und Java bekannt, wo sie genutzt werden, um Elemente im Code zu dekorieren,
z.B. Klasen und Methoden. Mit Attri\-bu\-ten kann eine Sprache erweitert werden, ohne dass der Compiler angepasst werden muss. Attribute werden verwendet, um die Semantik von dom\"{a}nespezifischen Abstraktionen (DSA) zu emulieren. Attributfamilien werden in dieser Arbeit verwendet, um Domainvariablit\"{a}t zu unterst\"{u}tzen. Eine leichtgewichtige  
Spracherweiterung wird vorgeschlagen, die Abfragen und Manipulation von attributiertem Code erlaubt. Diese Sprachen werden \textit{Generalized and Annotated Abstract Syntax Tree (GAAST)} Sprachen genannt. GAAST Sprachen erlauben auch bessere Sprachunterst\"{u}tzung f\"{u}r Model Driven Architecture (MDA) Transformationen.

Drittens, um den Aufwand f\"{u}r Attributbasierte DSA niedrig zu halten, wird eine strukturierte Modularisierung der attributgesteuerten Transformatoren ben\"{o}tigt. Spezialisierte attri\-but\-ba\-sierte Transfor\-ma\-to\-ren verbinden attributbasierte DSAs mit Containerdiensten f\"{u}r eine bestimmte Dom\"{a}ne. Attributgesteuerte Transformatoren sind horizontal modularisiert, gem\"{a}{\ss} den Eigenschaften der modellierten Dom\"{a}nenfunktionalit\"{a}t. Die Semantik der Funktionalit\"{a}t der Dom\"{a}ne wird genutzt, um Attributfamilien zu modellieren und um den Tranformationsfluss zu bestimmen. Ein einheitliches Kompositionsmodell f\"{u}r Trans\-for\-ma\-to\-ren, das auf inne\-ren Anno\-ta\-tionen beruht, wird entwickelt, um es den Transformatoren zu erm\"{o}glichen ihre Abh\"{a}ngigkeiten deklarativ zu definieren. Innere Annotation erlauben es den Transformatoren jedes Element das sie bearbeiten mit beliebigen Metadaten f\"{u}r nachfolgende Transformatoren zu versehen. Das fest verdrahtete Metamodell der object-orientierten Sprache wird benutzt, um eine vertikale Modularisierung der Attributgesteuerten Transformatoren zu erm\"{o}glichen. 
Die Transfor\-ma\-tions\-stra\-te\-gie ist gem\"{a}{\ss} der Hierarchieeben der Ele\-men\-te des Programmcodes strukturiert. Die Strategie wird erzwungen durch Operationen, die spezialisiert sind f\"{u}r bestimmte Ele\-me\-nte des Me\-ta\-mo\-dells der Sprache. Diese Operationen bilden eine spezialisierte Sprache f\"{u}r attributgesteuerte Transformatoren. Attributoperationen, wie zum Beispiel das \"{U}berpr\"{u}fen von Abh\"{a}ngigkeiten zwischen Attributen, werden dekarativ spezifiziert und aus den Transformatoren herausfaktorisiert durch die Benutzung von Meta-Attributen. Meta-Attribute erm\"{o}glichen die Dekoration von Attributen mit beliebiger Semantik, die automatisch verarbeitet werden k\"{o}nnen.  

Die beschriebene Technologie wird evaluiert anhand eines erweiterbaren
Frameworks f\"{u}r mobile Container; entwickelt  f\"{u}r  Anwendungen die das Mobile Information Device Profile (MIDP) der Java 2 Mobile Edition verwenden.
Verschiedene technische Belange von MIDP Anwendung werden adressiert, z.B.  
Persistenz, Sitzungs- und Darstellungsmanagement, und Netzwerkkommunikation. Die Verwendung mobiler Container resultiert in einem einfacheren und deklarativeren Programmiermodell, das die Architektur der Dom\"{a}ne in Anwendungen erh\"{a}lt.
